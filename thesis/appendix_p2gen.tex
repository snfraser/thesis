\section{Phase 2 model generator and its configuration}
\label{sect:app_p2gen}

The Phase 2 model generator can be used to create ODB snapshots according to various distributions and with varying degrees of complexity. The Phase 2 model is generated with the following parameters:-

\subsection{Notation}

The following distributions are used throughout:-

\begin{itemize}
\item $U[a,b]$ a uniform random number selected from the range $[a, b]$.
\item $G(\mu, \sigma)$ a random number with probability distributed as a gaussian with mean $\mu$ and standard deviation $\sigma$. 
\end{itemize}

\subsection{Configuration parameters}
\begin{description}
\item [Root] Specifies the root name for this phase2 model. This is the name by which the model is accessed via a Phase2ModelProvider.
\item [Name] The name of the site.
\item [$\Phi$] Latitude of the site.
\item [$L$] Longitude of the site.
\item [$T$]  specifies the \emph{current} point in the semester. Typically this time coincides with the start time of a simulation run.
\item [$\Delta{T_B}$] represents the time difference from $T$ to the start of the current semester.
\item [$\Delta{T_F}$] represents the time difference from $T$ to the end of the current semester.
\item [$N_P$] controls the total number of proposals generated. These will have activation times distributed according to $U[\Delta{T_B}, T]$ and expiry dated distributed according to  $U[T, \Delta{T_F}]$.
\item [$N_G$] controls the number of groups per proposal, distributed as $U[1, N_G]$.
\item [$N_O$] controls the number of observations per group, distributed as $U[1, N_O]$.
\end{description}

Groups are generated using various timing constraint classes according to the following fractions:-
\begin{description}
\item [$g_{flex}$] Fraction of groups to be generated with Flexible timing constraints.
\item [$g_{mon}$]  Fraction of groups to be generated with monitor timing constraints.
\item [$g_{int}$]  Fraction of groups to be generated with minimum interval timing constraints.
\end{description}

All groups are constructed with start and expiry dates selected from the range $U[\Delta{T_B}, T]$ and $U[T, \Delta{T_F}]$ respectively. 

For monitor groups, the periods are selected with probability determined by variable $f_{m_i}$ from one of a set of 3 normal distributions $G(\mu_{m_i}, \sigma_{m_i})$ defined by variables:-

\begin{description}
\item [$\mu_{m_i}$]     mean value for monitor period distribution $i$.
\item [$\sigma_{m_i}$]  standard-deviation for monitor period distribution $i$
\end{description}
The window fraction is taken from the distribution $U[0.25, 1.0]$.


For MinimumInterval groups, the minimum window is selected with probability $f_{i_i}$ from one of 3 normal distributions $G(\mu_{i_i}, \sigma_{i_i})$ defined by variables:-

\begin{description}
\item [$\mu_{i_i}$]     mean value for minimum interval length distribution $i$.
\item [$\sigma_{i_i}$]  standard-deviation for minimum interval lengthdistribution $i$
\end{description}


Observation exposure times are selected with probability $e_i$ from one of 2 exposure length distributions $E(\mu_{e_i}, \sigma_{e_i})$ defined by variables:-
\begin{description}
\item [$\mu_{e_i}$]    mean value for exposure length distribution $i$.
\item [$\sigma_{e_i}$] standard-deviation for exposure length distribution $i$.
\end{description}
$N_n$ exposure-max-count specifies the maximum number of multruns per observation, the number of multruns is taken from $U[1,N_m]$. 


The budget available to each proposal is taken from $U[B_{min}, B_{max}]$ with used fraction distributed as $U[U_{min}, U_{max}]$.

Proposals have a scientific priority rating taken from the range 0 (HIGH) to 3 (LOW). The allocation is specified accroding to the fractions:-
\begin{description}
\item [$P_0$] Fraction of proposals with scientific priority 0 (HIGH).
\item [$P_1$] Fraction of proposals with scientific priority 0 (MED).
\item [$P_2$] Fraction of proposals with scientific priority 0 (LOW).
\end{description}

Groups have internal priorities selected from the following set.
\begin{description}
\item [$G_1$] Fraction of groups with priority level 1 (normal)
\item [$G_2$] Fraction of groups with priority level 2 (raised)
\item [$G_3$] Fraction of groups with priority level 3 (medium)
\item [$G_4$] Fraction of groups with priority level 4 (high)
\item [$G_5$] Fraction of groups with priority level 5 (urgent)
\item [$G_{BGR}$] Fraction of groups with Background priority level
\item [$G_{STD}$] Fraction of groups which represent photometric standards (nominal priority 3).
\end{description}

Various observing constraints are represented by the following parameters:-
\begin{description}
\item [$f_{Dark}$] Fraction of groups which require Dark lunar conditions.
\item [$f_{Photom}$] Fraction of groups which require photometric extinction conditions.
\item [$f_{Poor}$] Fraction of groups which require minimum seeing of \emph{poor} ($ seeing > 1.3^{\prime\prime}$).
\item [$f_{Av}$] Fraction of groups which require minimum seeing of \emph{average} ($ 0.8^{\prime\prime} < seeing < 1.3^{\prime\prime} $).
\item [$f_{Ex}$] Fraction of groups which require minimum seeing of \emph{excellent} ($ seeing < 0.8^{\prime\prime}$).
\end{description}


 
 

