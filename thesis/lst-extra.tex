So far we have been looking simple simulations with fixed E,P models and allowing the only source of random variation to be the SEM. Operationally it is found that the actual exec times of groups is modelled reasonably well by the Exec timing model so this would infact account for a small effect.

We now wish to see how effective the various schedulers are under varying P and E models. In particular we would like to be able to obtain a single characteristic for a P or E model which would allow us to plot scheduler  SQMs against these characteristics.

In reality to fully describe an E or P model requires a number (in the case of P, a very large number) of parameters. We therefore seek to find a single charcateristic which can be measured for these models and thus sued a s an independant variable.

From a defined set of E model parameters we can generate any number of E scenarios (effectively instantiations of the model) - it would be useful to see if we get variation in results and indeed in the ``measured characteristic'' (SQMs) from different scenarios generated by the same model - if this variation is large we will have to run many more simulations and there will be x-errors in addition to the expected y-errors.

A similar situation occurs with P models - we can generate many P scenarios (instantitations) from a single set of P model parameters - and hence probably different ``measured characteristics''.

We need to check the stability of characteristics for initial model parameters.

Using an E model with stability parameter tau such that the seeing remains in a stable state for a period averaging around tau. The seeing scenario is generated by taking random intervals calculated from -Tau*ln(1-R) where R is a random number in [0,1]. At each period the seeing changes randomly to another value - we are not bothered which value or how the seeing states are distributed only that it is changing. Results of a run with Tau=30mins are shown in Fig.~\ref{xxx} which shows the actual seeing scenario over a period of XX days and Fig.~\ref{xxx} which shows the relative and cumulative distribution of lengths of periods of stability. For comparison scenarios with Tau=1h and 2h are also shown