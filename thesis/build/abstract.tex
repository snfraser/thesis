\begin{abstract}
Astronomical telescopes operate in a dynamic and uncertain operational environment. 

Time is often oversubscribed with programs competing for available slots and the best observing conditions. 

In order to achieve a high scientific return a scheduler must be able to create a globally optimal observing schedule. In dynamic environments any offline or static schedule will become repidly out-of-date, thus iterative repair based or dynamic despatch scheduling is the preferred option being able to respond quickly to such changes. Dynamic despatch suffers from myopism, long term, global aims are usurped in favor of local optimization - what is best to do right now. 

%I show that by incorporating a predictive element into a scheduler by using look-ahead planning to generate medium term objectives to be followed and short term environmental prediction to attempt to keep local decisions close to the projected plan. 

I investigate and characterize the operating environment and investigate techniques for short term prediction leading to the design of an architecture for building schedulers from a series of components. I investigate methods for characterizing the value of schedules (metrics) both locally (at the decision point) and over longer horizons. An extensible software architecture  is designed incorporating a simulation framework to allow a variety of schedulers to be implemented. Experiments are performed using the scheduler component architecture and simulation framework to determine the effects on schedule quality of environmental stability, disruptive events and reliability of prediction under a range of load conditions. These show that where conditions are relatively stable there is an advantage to using longer look-ahead horizons but that where conditions are unstable a basic despatch scheduler is as good or better. Further experiments show that look-ahead schedulers can achieve better quality than a despatcher when the loading of the schedule is high.




%The scientific objectives and conditions for optimal execution of individual observing programs resides naturally with the program - ways of trading time slots and conditions may be peculiar to a given program, this suggests we look at ways of distributing the decision-making between delegates (agents) representing the value judgements of the observations' owners (PIs). The task of scheduling, distributed between these agents then emerges from their interaction under the control of a supervisor to ensure global aims are retained. 
\end{abstract}
