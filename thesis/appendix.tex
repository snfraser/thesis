\appendix

\newpage
\section{Phase2 Database}
\label{sect:app_phase2db}
Here describe the P2DB and the objects in it and how they fit into the architecture. Define what they are and how they are used/modified by scheduler.

\subsection{Proposal}
A Proposal contains a number of groups, a list of targets for the observations in the groups to observe and some linking information which specifies temporal or other links between groups.

\subsection{Groups}
A group is a set of observations which is performed as a single scheduling entity - atomic. A group contains a list of observations, timing/enablement constraints, observing condition constraints/preferences, sequencing information for the observations and a quality-of-service metric (i.e. a class which can examine the execution results and yield a QOS measurement). 
Note: The QosMetric may just be represented by a set of user-specified preferences which are fed into a default QosMetric class to get some sort of QOS measure.

NOtes about constraints:-
Some are deterministic others are variable and uncertain - un-predictable or poorly predictable as horizon increases.
\begin{description}
\item[Timing constraints]
Always predictable. Various categories - flexible (one-off), monitoring (cyclic/repeating), fixed/ephem (need special treatment)  
\item[Observing constraints]
Deterministic - (moon distance, airmass limit).
Variable - (seeing limit, photom limit).
\end{description}

Group made up of observations.

Utility functions - aggregation scores or averaging for target parameters.
list of scoring metrics. 

\subsection{Observations}
What is involved - slew, rotator alignment, instrument config then series of exposures and offsets.
Executed by RCS - it decomposes into tasks and handles error recovery if possible.

